\documentclass[12pt,fleqn]{article}

\usepackage[cp1251, utf8x]{inputenc}
\usepackage[T2A]{fontenc}
\usepackage{amssymb, amsmath, mathrsfs, amsthm}
\usepackage[russian]{babel}
\usepackage[footnotesize]{caption2}
\usepackage{indentfirst}
\usepackage{graphicx}
\usepackage[noend]{algorithmic}
\usepackage{algorithm}
\usepackage{color}
\usepackage{hyperref}
\usepackage{array}
    
\hypersetup{
	colorlinks = true,
	urlcolor = blue
}

\newcolumntype{L}[1]{>{\raggedright\let\newline\\\arraybackslash\hspace{0pt}}m{#1}}
\newcolumntype{C}[1]{>{\centering\let\newline\\\arraybackslash\hspace{0pt}}m{#1}}
\newcolumntype{R}[1]{>{\raggedleft\let\newline\\\arraybackslash\hspace{0pt}}m{#1}}

\begin{document}
\begin{titlepage}
\begin{center}
    Московский государственный университет имени М. В. Ломоносова

    \bigskip
    \includegraphics[width=50mm]{msu.eps}

    \bigskip
    Факультет Вычислительной математики и Кибернетики\\
  	Кафедра Математических Методов Прогнозирования\\[10mm]

	\textsf{\large\bfseries
        Обработка и методы распознавания изображений
    }\\[10mm]
    \textsf{\large\bfseries
        Отчет по лабораторной работе №2
    }\\[10mm]
	
	\bigskip
	\bigskip
	\bigskip
	\bigskip
	\bigskip
	\bigskip
	\bigskip
	\bigskip
	\bigskip
	
    \begin{flushright}
        \parbox{0.5\textwidth}{
            Выполнил:\\
            студент 3 курса 317 группы\\
            \emph{Таскынов Ануар}\\[5mm]
        }
    \end{flushright}

    \vspace{\fill}
    Москва, 2016
\end{center}
\end{titlepage}

\newpage
\renewcommand{\contentsname}{Содержание.}
\tableofcontents
\newpage
\section{Постановка задания.}

В данном задании необходимо было разработать программу для классификации изображений ладоней, по сгенерированным признакам. В качестве признакового описания строилась <<линия пальцев>>, соединяющая основания пальцев с их кончиками. 

Язык программирования Python 2. Эксперименты приведены в Ipython Notebook'е, функции описаны в отдельном python-модуле. Выполнен уровень Expert.

В уровне Expert помимо генерации признаков, необходимо было провести кластеризацию изображений.

\newpage
\section{Описание решения.}
\subsection{Бинаризация.}
Первым делом необходимо бинаризовать изображение. Это делалось следующим образом:
\begin{itemize}
\item Проводилась бинаризация по методу Оцу.
\item Найденный порог $t$ умножался на $1.6$. Это было сделано для того, чтобы полностью отделить руку от фона, так как некоторые изображения при пороге $t$ отделялись не полностью, например рис \hyperref[Image1]{1}
\end{itemize}


\begin{figure}[h]
\begin{minipage}[h]{0.49\linewidth}
\center{\includegraphics[width=0.95\linewidth]{bad_binary12} \\} 
\end{minipage}
\hfill
\begin{minipage}[h]{0.49\linewidth}
\center{\includegraphics[width=0.95\linewidth]{bad_binary13} \\ RRF} 
\end{minipage}
\caption{Изображения, на которых плохо отработал порог $t$.}
\label{Image1}
\end{figure}

\section{Результаты кластеризации.}
\begin{table}[h]
\begin{center}
\begin{tabular}{|c|c|c|c|c|c|}
\hline
Имя образца & Соседи & Имя образца & Соседи & Имя образца & Соседи \\ \hline
         001 &            002 037 090  &         039 &            037 002 001  &         095 &            109 008 067  \\ 
         002 &            001 037 145  &         041 &            060 105 049  &         096 &            063 093 031  \\
        003 &            006 007 005  &         046 &            020 018 016  &         097 &            007 003 014  \\
        004 &            006 003 007  &         047 &            050 060 146  &         099 &            012 014 003  \\
         005 &            007 003 006  &         049 &            047 060 041  &         105 &            142 107 106  \\
        006 &            003 004 007  &         050 &            047 060 146  &         106 &            142 105 107  \\
         007 &            005 003 155  &         051 &            052 054 053  &         107 &            105 088 091  \\
        008 &            067 066 057  &         052 &            051 054 053  &         109 &            092 067 113  \\
         009 &            011 010 065  &         053 &            054 052 051  &         111 &            096 031 063  \\
        010 &            009 011 065  &         054 &            051 052 078  &         112 &            114 113 092  \\
         011 &            079 009 081  &         055 &            053 052 106  &         113 &            114 112 092  \\
         012 &            014 013 006  &         056 &            086 076 057  &         114 &            112 113 092  \\
         013 &            012 014 015  &         057 &            076 056 081  &         118 &            123 122 113  \\
         014 &            012 013 006  &         060 &            047 145 050  &         120 &            124 034 028  \\
        015 &            013 005 014  &         063 &            096 034 066  &         122 &            123 092 026  \\
         016 &            017 020 046  &         064 &            141 066 009  &         123 &            122 118 092  \\
        017 &            016 020 046  &         065 &            009 010 155  &         124 &            034 120 028  \\
         018 &            021 019 020  &         066 &            128 129 011  &         126 &            127 124 120  \\
         019 &            021 018 078  &         067 &            008 109 129  &         127 &            126 122 123  \\
        020 &            046 018 016  &         068 &            020 011 079  &         128 &            129 066 079  \\
        021 &            019 018 020  &         071 &            150 151 050  &         129 &            128 066 008  \\
        022 &            023 035 026  &         076 &            077 079 086  &         135 &            003 157 107  \\
         023 &            022 035 124  &         077 &            076 079 086  &         138 &            141 065 064  \\
         024 &            086 021 079  &         078 &            077 076 079  &         141 &            009 064 065  \\
        026 &            028 029 022  &         079 &            076 077 011  &         142 &            105 106 107  \\
         027 &            029 028 036  &         081 &            079 086 082  &         144 &            004 145 146  \\
         028 &            026 124 029  &         082 &            090 081 007  &         145 &            144 146 002  \\
         029 &            027 028 026  &         086 &            077 079 076  &         146 &            004 144 006  \\
         031 &            063 096 009  &         088 &            091 107 157  &         150 &            151 152 071  \\
         034 &            124 120 028  &         090 &            082 005 007  &         151 &            150 152 071  \\
         035 &            022 023 026  &         091 &            088 107 105  &         152 &            150 151 071  \\
         036 &            035 120 028  &         092 &            112 114 113  &         155 &            007 006 005  \\
         037 &            002 001 039  &         093 &            096 063 092  &         157 &            107 086 155  \\ \hline
\end{tabular}
\caption{Таблица ближайших соседей.}
\label{tabular1}
\end{center}
\end{table}


\begin{table}[H]
\begin{center}
\begin{tabular}{|c|l|}
\toprule
\hline
Человек &               Изображения ладоней \\ \hline
\midrule
       1 &              012 013 014 015 097  \\
       2 &                      026 028 122  \\
       3 &                      150 151 152  \\
       4 &                  051 052 053 054  \\
       5 &                      066 128 129  \\
       6 &                  092 112 113 114  \\
       7 &                              049  \\
       8 &                  009 010 065 141  \\
       9 &                      144 145 146  \\
      10 &              016 017 020 046 068  \\
      11 &                  001 002 037 039  \\
      12 &                              111  \\
      13 &                      008 067 109  \\
      14 &                      031 063 096  \\
      15 &                  105 106 107 142  \\
      16 &                          027 029  \\
      17 &                              138  \\
      18 &                  022 023 035 036  \\
      19 &                          126 127  \\
      20 &  011 056 057 076 077 079 081 086  \\
      21 &                              071  \\
      22 &  003 004 005 006 007 099 155 157  \\
      23 &                      047 050 060  \\
      24 &                              064  \\
      25 &                          118 123  \\
      26 &                      034 120 124  \\
      27 &                              095  \\
      28 &                              135  \\
      29 &                      055 082 090  \\
      30 &                              041  \\
      31 &                          088 091  \\
      32 &                              093  \\
      33 &              018 019 021 024 078  \\

\hline
\bottomrule
\end{tabular}
\caption{Таблица кластеров.}
\label{tabular2}
\end{center}
\end{table}


\end{document}
